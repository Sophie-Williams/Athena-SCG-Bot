% vim: set textwidth=75 :
\documentclass[letterpaper,12pt,oneside]{article}

%\usepackage{fullpage}

%\thispagestyle{empty}
%\pagestyle{empty}

\usepackage{amsmath}
\usepackage{amssymb}

\title{Athena}
\author{Alex Lee, William Nowak}
\date{\today}

% No longer than 7 pages
\begin{document}
\maketitle

% It is important that your code itself is appropriately documented.
% In the project document you reflect on your project experience
% and tell the story behind your project. 

% There are two parts.

% Part 1. Design Decisions 
% ========================
% highlights the difficulties you had and motivates
% your important design decisions. It also compares your original 
% road map with the actual road taken by your project.
\section{Design Decisions}

Not using Java was the best decision we made. We both think of Java as a
lesser language than Python.

The MAX-CSP problem is exponential in nature. So, In the beginning
stages, it was beneficial to have speed while searching for a better
algorithm. At one point, we were doing five million random solves. This
was only possible because we were using C.


% Part 2. Innovation
% ==================
% In this course you developed an agent that can fend for itself
% against the agents developed by your peers.
% You developed your agent with great freedom:
% You could choose the programming language, implementation technologies,
% architecture of your agent, algorithms etc. The only requirement was to win
% the game. At the same time, you collaborated with your
% peers by giving them feedback that helped them to locate faults
% in their code.
\section{Innovation}

\subsection{Freedom}
% How did you take advantage of the freedom you had?
% Did the freedom inspire you to investigate more options?
% Describe the options you considered and which ones
% you implemented.

For the first few competitions, we leveraged the fact that,
\begin{itemize}
\item C is faster than Java
\item Python has a nice interface to C
\end{itemize}
Our solver searched every possible solution where it would be impossible to
do in reasonable time in Java.

At one point, we discussed having a networked solver, which would
distribute work over many machines. However, it was too much work with
little payoff.

\subsection{Competitive / Cooperative Nature}
% The game is competitive / cooperative.
% How did you react to the competitive nature of the game?

The competitive nature was a huge driving factor in getting us to write
the actual code. It was not, however, a factor in producing {\em good}
code.

There was not much cooperation. Will's competition runner and Brent's
history viewer are nice tools, but they ultimately did not affect our
agent in a significant way.

\subsection{The need to do all steps well}
% Writing winning agents takes a lot of skill: at the conceptual,
% design and implementation level.
% If the conceptual understanding is perfect, but there is a flaw 
% in the design or implementation, your agent probably won't win. 
% How did you react to the need to do all steps well.

Writing the parser and creating the infrastructure (the straight-forward
parts that everyone can do) had virtually nothing to do with winning. All
the components that actually mattered were strenuous exercises in algorithm
design.

\subsection{Diversity}
% The success of such a class depends on having a community
% of students with different skills: Some like a more
% tactical approach, others strive for a strategic solution.
% Some are good programmers and others are not but they have other skills.
% Some like to use new tools and others only well tested tools.
% 
% How did you react to the diversity that was all focused through the game.

As the minority group, we did not encounter real diversity.

\subsection{Grades}
% How did you react to using the competition results as part of your grade?

We would not have minded so much if the code for the administrator was not
the specification.

\subsection{Feedback}
% Did the feedback you received through the competitions help
% you find bugs in your software?

\subsection{Observe, Design, Implement, Test}
% The other agents posed problems to you. Did the cycle:
% Observe history - Identify issues - Plan an approach - 
% Design your code - Implement - Test
% that you were exposed to after every competition help 
% you improve your problem solving skills?
There were no data mining efforts that made into the code base. Everything
was designed, implemented, then tested.

We did look at the history files of the games where we lost, but we didn't
let it sway how we implemented the code.

\subsection{Problems create by your peers}
% Many software design and implementation problems you had to solve during the course
% were caused by your peers. I only made sure through the game design
% that they could pose reasonable problems to you.
% Did the fact that you solved problems created by your peers
% motivate you?

\subsection{SCG skills}
% The SCG game was designed so that it is sound, i.e.,
% the game score reflects **only** the agent's skill level in:

\subsubsection{Solving problems}

This skill is seldom tested, since the game itself is not a benchmark on
the solver alone. There have been games were no problems were solved. There
have been games were only one agent solved. So, even logically, the game is
not a reflection on the solver.

\subsubsection{Providing hard problems}

Same thing for this skill. There can be an instance of a game were nothing
is ever provided (only offers and re-offers).

\subsubsection{Introspective skills}

% Argue for or against soundness of the SCG game.

\subsection{SCG game}
% 2.9
% The game was designed so that it has the following 3 properties:

\subsubsection{An agent can't force other agents to consistently lose}
\subsubsection{An agent can't force other agents to consistently draw}
\subsubsection{An agent can't force other agents to consistently win}

% Argue for or against those 3 properties of the SCG game.
% 
% The entire document should not be longer than 7 pages. 
% 
\end{document}
